
\documentclass[compress]{beamer}


\usepackage[utf8]{inputenc}
\usepackage[T1]{fontenc}
\usepackage{dwlecture}
\usepackage{tikz}


\subject{Testing Java ME applications}

\title[Automated Integration Tests for Mobile~Applications in J2ME]%
{Automated Integration Tests for Mobile~Applications in Java~2~Micro~Edition}

\author{Marcin Zduniak}

\institute{Institute of Computing Science\\Poznan University of Technology}

\date{2007}

% \AtBeginSection[] {
%   \begin{frame}<beamer>
%     \framesubtitle{\insertpart}
%     \tableofcontents[sectionstyle=show/shaded,subsectionstyle=show/shaded/hide]
%   \end{frame}
% }
% 
% \AtBeginSubsection[] {
%   \begin{frame}<beamer>
%     \framesubtitle{\insertpart}
%     \tableofcontents[sectionstyle=shaded,subsectionstyle=show/shaded/hide]
%   \end{frame}
% }


\begin{document}

\begin{frame}[plain]
    \titlepage

    % Place the logo and gradient box.
    \xygraphics{11}{0.5}{right}{bottom}{width=1cm}{figures/template/logo-pp}
    \begin{textblock}{1}(0,9.6)
        \begin{tikzpicture}
            \useasboundingbox (0,0) rectangle (12.8,-9.6);
            \shade[bottom color=blue!20!white,top color=white,shading=axis,shading angle=90] 
                (11,1.5) rectangle (12.8,0.5);
        \end{tikzpicture}
    \end{textblock}
\end{frame}

\section{Participants}  % --------------------------------------------------------------------------

\begin{frame}
  \frametitle{Who is who}

  \begin{itemize}
      \item dr inż.~Bartosz Walter
      \item dr inż.~Dawid Weiss
      \item inż.~Marcin Zduniak
  \end{itemize}
\end{frame}

\section{Motivation} % -----------------------------------------------------------------------------

\subsection[Test types]{Types of Application Tests} % ----------------------------------------------

\begin{frame}

    \begin{center}\Large\bfseries
    What are ``software tests''?
    \end{center}

    \pause\bigskip
	\begin{block}{(Wikipedia)}
        \emph{Software testing} is the process used to help identify the \emph{correctness}, 
        \emph{completeness}, \emph{security}, and \emph{quality} of developed computer software.
	\end{block}
\end{frame}

\begin{frame}
    \only<1>{
        \begin{center}\Large\bfseries
        How can we ``test software''?
        \end{center}
    }

    \only<2->{
        {\color{blue} Unit tests}\\
        Correctness of individual units of source code

        \medskip
        {\color{blue} Module/ integration tests}\\
        Chunks of functionality, sometimes the entire program.
        testing in various target environments (O/S's, processors etc.).

        \medskip
        {\color{blue} Acceptance tests}\\
        Compliance to customer's requirements; often manual work.

        \medskip
        {\color{blue} Regression tests}\\
        System stability/ correctness in response to ongoing changes.
    }
\end{frame}


\subsection[Scope]{Project scope} % ---------------------------------------------------------------- 

\begin{frame}
    \frametitle{Project scope}

	\begin{itemize}
        \item Tool for acceptance, integration and regression tests for Java ME business 
            applications.
        \item Automatic (capture-replay).
        \item Works on both actual devices and emulators.
        \item XML testing scripts compiled to binary intermediate language and vice versa.
	\end{itemize}
\end{frame}

\begin{frame}[plain]
    \frametitle{Usage example}

    \only<1>{
        \hgraphics{figures/process1}
        \lbcaption{Scenario with capture-replay process.}
    }

    \only<2>{
        \hgraphics{figures/process2}
        \lbcaption{Scenario with human-friendly test scripts.}
    }
\end{frame}

\subsection[J2ME]{Writing and testing applications in J2ME} % --------------------------------------

\begin{frame}
    \frametitle{Constraints}

	\begin{itemize}
        \item Programming and testing are more difficult compared to desktop programs.
        \item Each mobile has different hardware configuration
        \item `Standard' APIs implemented by hardware vendors contain differences.
        \item A number of non-standard APIs and proprietary solutions.
        \item Lack of system-level support for handling events (GUI, sounds, SMS, \ldots{}).
	\end{itemize}
\end{frame}

\subsection[Related]{Related works} % --------------------------------------------------------------

\begin{frame}
    \frametitle{Related projects}

	\begin{itemize}
        \item J2ME Unit
        \item Sony-Ericsson Mobile JUnit
        \item Motorola Gatling
        \item CLDC Unit
        \item IBM Rational Test RT
        \item Research In Motion -- BlackBerry Fledge emulator
	\end{itemize}
\end{frame}


\section{Implementation} % -------------------------------------------------------------------------

\subsection[Code injection]{Dynamic code injection} % ----------------------------------------------

\begin{frame}
    \frametitle{Dynamic code injection}

	\begin{itemize}
        \item One possible solution for dealing with environment's constraints.
        \item Intercepting \emph{events} and injecting our custom \emph{proxies}
            at several \emph{injection points}.
	\end{itemize}
\end{frame}

\begin{frame}[containsverbatim,plain]
    \rbcaption{Code injection example -- capturing \texttt{startApp} event.}

\begin{javablock}
public final class MyMidlet extends MIDlet {
    protected void startApp() 
        throws MIDletStateChangeException {
        // original code
    }
...
\end{javablock}

\begin{javablock}
public final class MyMidlet extends MIDlet {
    protected void startApp() 
        throws MIDletStateChangeException {
        // record: before-start-event
        try {
            this.orig$startApp();
            // record: after-start-event
        } catch (Throwable t) {
            // record: start-exception-event
        }
    }

    private void orig$startApp()
        throws MIDletStateChangeException {
        // original code
    }
...
\end{javablock}

\end{frame}

\subsection{Testing lifecycle} % -------------------------------------------------------------------

\begin{frame}
    \frametitle{Phase 1: Recording}

	\begin{itemize}
        \item Intercepting user events (key press, pointer events, text edit).
        \item Intercepting environment events (sounds, screen changes, Internet access, SMS).
        \item Transferring to remote console (through GPRS, bluetooth, IRDA or serial port).
	\end{itemize}
\end{frame}

\begin{frame}
    \frametitle{Phase 2: Replaying}

	\begin{itemize}
        \item Simulating user events (key press, pointer events, text edit).
        \item Assertions: Intercepting environment events (sounds, screen changes, Internet access, SMS).
        \item Transferring to remote console (through GPRS, bluetooth, IRDA or serial port).
	\end{itemize}
\end{frame}

\begin{frame}[containsverbatim]
    \frametitle{Phase 3: Test maintenance}

    Maintenance through \emph{human-comprehensible test scripts}.

    \bigskip
\begin{xmlblock}
<scenario>
  <event timestamp="1000">
      <displayable-changed title="Hello screen" type="TEXTBOX" />
  </event>
    
  <event timestamp="2000">
      <command cmdLabel="Start app" displayableTitle="Hello screen" />
  </event>
    
  <event timestamp="3000">        
      <textbox-modification assertion="true" strongAssertion="true" 
                            string="I like testing" />
  </event>
</scenario>
\end{xmlblock}

\end{frame}


\subsection[Prototype]{Proof-of-concept prototype} % -----------------------------------------------

\begin{frame}[plain]
	\frametitle{Screenshots from test recording session.}

    \lbcaption{Server console.}
    \rbcaption{Emulator window.}

    \fxygraphics{12.4}{0.5}{right}{bottom}{width=5cm}{figures/1}
    \fxygraphics{0.2}{8.4}{left}{top}{width=12.4cm}{figures/2}

\end{frame}


\section*{Summary} % -------------------------------------------------------------------------------

\begin{frame}[plain]
    \frametitle{Summary}
    
    \begin{itemize}
        \item Useful testing framework for J2ME environment \emph{does not exist}.
        \item Solution: \emph{dynamic code injection}.
        \item For \emph{developers} and \emph{clients}.
    \end{itemize}
\end{frame}

\begin{frame}[plain]
    \frametitle{Little victories}
    
    \begin{itemize}
        \item Poznan University of Economics -- `10th International Conference on Business Information Systems'
        \item UAM Foundation -- `Pomysł na biznes' competition
    \end{itemize}
\end{frame}

\begin{frame}[plain]
	\begin{center}
	Thank you for your attention.
	\end{center}
\end{frame}

\end{document}
