\chapter{Summary and Conclusions}

\section{Summary}

% Remind the reader what was planned. Sum up what's been achieved and what hasn't (and why)?

This thesis is a proof of concept demonstrating that fully-fledged capture-replay
testing framework is feasible in Java 2 Micro Edition environment. The
prototype implementation has been written for it and showed usable in practice. 
The framework may be used both in fully automated testing environments
exploiting mobile device simulators and in real environments with real
mobile devices.

Parts of this thesis have been published at the 10th International Conference on 
Business Information Systems, Poznan, Poland~\cite{zduniak-weiss-2007}.


\section{Future Directions}

Our biggest challenge at the moment is to provide some objective means to
assess the value gained from using the framework. What common sense states
as obvious is quite difficult to express with hard numbers. We considered a controlled
user experiment where participants would be given the same application
and a set of tasks to implement (refactorings and new features). Half of the
group would have access to the results of integration tests, the other half would
just work with the code. We hoped this could demonstrate certain gains (number
of early detected bugs, for example) that eventually translate into economic
value for a company. Unfortunately, this kind of experiment is quite difficult to
perform and its results are always disputable (i.e., due to ranging skills between
programmers), so we decided to temporarily postpone it. Other possible research
and technical directions are:
\begin{itemize}
    \item Design a flexible architecture adding support for events that are outside
the scope of the J2ME specification, but are commonly used in mobile development.
These events include, for example, vendor-specific APIs such as
vibration or backlight provided by Nokia.
    \item Implement alternative event serialization channels –- through serial cables or
Bluetooth connections.
    \item Consider evaluation schemes for the presented solution. A real feedback from
developers translates into a proof of utilitarian value of the concept -– does
the testing framework help? How much time/ work does it save? What is
the ratio of time spent on recording/ correcting test scripts compared to
running them manually? We should emphasize that these questions are just
as important as they are difficult to answer in a real production environment.
    \item Integrate the framework with popular integrated development environments.
This goal is very important because developers must be comfortable with the
tool to use it and must feel the benefits it provides. Instant hands-on testing
toolkit would certainly assimilate faster in the community than an obscure
tool (such as Sony’s).
    \item We also think about extending the concepts presented in this paper to other
Java-based platforms for building mobile applications, such as NTT DoCoMo
Java, BlackBerry RIM API or Qualcomm Brew. They may not be as popular
as J2ME, but the concepts we have presented should be applicable in their
case as well.
    \item Based upon the experience and feedback from the initial prototype, we asses there is a real demand
    and market need for such a testing tool. Writing a production-quality utility from
    scratch (that developers and testers could use with little knowledge of the internal mechanisms) 
    is a promising direction  for future work.
\end{itemize}
