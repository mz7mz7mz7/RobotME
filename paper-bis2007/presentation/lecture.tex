
\documentclass[compress,notes=hide]{beamer}


\usepackage[utf8]{inputenc}
\usepackage[T1]{fontenc}
\usepackage{dwlecture}
\usepackage{tikz}

\usepackage{lmodern}
\usepackage[scaled]{uarial}


\subject{Testing Java ME applications}

\title[Automated Integration Tests for Mobile~Applications in J2ME]%
{Automated Integration Tests for Mobile~Applications in Java~2~Micro~Edition}

\author{Dawid Weiss, \textbf{Marcin Zduniak}}

\institute{Institute of Computing Science\\Poznan University of Technology}

\date{2007}

% \AtBeginSection[] {
%   \begin{frame}<beamer>
%     \framesubtitle{\insertpart}
%     \tableofcontents[sectionstyle=show/shaded,subsectionstyle=show/shaded/hide]
%   \end{frame}
% }
% 
% \AtBeginSubsection[] {
%   \begin{frame}<beamer>
%     \framesubtitle{\insertpart}
%     \tableofcontents[sectionstyle=shaded,subsectionstyle=show/shaded/hide]
%   \end{frame}
% }


\begin{document}

\begin{frame}[plain]
    \titlepage

    \pplogo
    
    \note{\footnotesize%
    Hello everyone, my name is Marcin Zduniak, I am a grad student at Poznan University of
    Technology and I'm here to present the initial results of my Master's thesis.

    \medskip
    The subject of my work was a proof-of-concept -- to design and implement a semi-automatic
    capture-replay testing environment for mobile applications written for the Java 2 Microedition
    environment.

    \medskip
    I would like to share with you the difficulties of this task and the approaches in
    which we solved them. First of all, let me give you an outline of the motivation and 
    scope of our work.
    }
\end{frame}

\section{Motivation}  % ----------------------------------------------------------------------------

\begin{frame}
  \frametitle{Introduction: software testing}

    \emph{Software testing} is the process used to help identify the correctness, completeness, 
    security, and quality of developed computer software.

    \bigskip
    \begin{columns}[T,onlytextwidth]
        \begin{column}{5cm}
            \begin{exampleblock}{Test type \raisebox{3pt}{\tikz \draw [->,thick] (0,0) -- (2,0);}}
                \begin{itemize}
                    \item unit tests
                    \item integration tests
                    \item \textbf{acceptance testing}
                    \item \textbf{regression tests}
                \end{itemize}
            \end{exampleblock}
        \end{column}
        \begin{column}{5cm}
            \begin{exampleblock}{Covered aspect}
                \begin{itemize}
                    \item single-developer, classes
                    \item cross-developer, modules
                    \item client's requirements
                    \item refactorings, bug fixes, stability
                \end{itemize}
            \end{exampleblock}
        \end{column}
    \end{columns}
    
    \note{
    There are many categories of software tests, as shown on the slide. Some concentrate on 
    verifying correctness of atomic units of the software (such as unit tests), 
    some on assuring proper high-level functionality (acceptance tests) and yet another
    class on providing quality assurance to the already written software -- these
    are regression tests. 
    
    \medskip
    In case of business mobile applications (we generally consider this to be any form-based
    application, with many screens and transitions between them), a convenient way
    to implement regressions tests is through \textbf{capturing} the interaction
    between the program and the user and \textbf{replaying} this interaction during
    testing, checking if the output is identical as before.

    \medskip
    Let me quickly demonstrate what this procedure is about.
    }
\end{frame}

\begin{frame}[plain]
    \rbcaption{Capture-replay regression test scenario.}
    \hgraphics{figures/process-outline-2}
    
    \note{
    Note that at the beginning the set of features planned for the release is a base
    for preparing test scripts. Very often these scripts are informal and are a result
    of the user's expectations.
    
    \medskip
    After (or better: during) the software is developed, the informal scenarios are checked
    against a candidate release. If the scenarios are fulfilled we release the software. Note
    that during this checking the interaction with the program can be recorded and later reused 
    to verify if the software behaves exactly as it used to during the initial release. This process
    can be automated fairly easily for desktop and web applications, but turns out to be
    quite difficult for mobile applications.
    }
\end{frame}

\begin{frame}
    \frametitle{Motivation}

    Programming and testing in J2ME is \textbf{more difficult} compared to 
    desktop programs:
    \begin{itemize}
        \item each device has slightly different hardware configuration,
        \item ``standard'' APIs from various vendors differ,
        \item a number of non-standard APIs and proprietary solutions exists.
    \end{itemize}
    
    \medskip
    Java 2 Mobile Edition \textbf{has no support} for fundamental facilities required
    to implement a capture-replay testing environment:
    \begin{itemize}
        \item lack of system-level support for handling events (GUI, sounds, SMS, \ldots{}),
        \item lack of system-level support for simulating events.
    \end{itemize}

    \note{
    In reality, mobile application development stops at the level of unit tests because 
    of lack of tools and techniques that would allow proper acceptance and regression tests.
    This slide presents a list of reasons why this is the case.
    
    \medskip
    First, in spite of the API specification, there is a wide range of hardware and software
    vendors. This leads to various incompatibilities and quirks -- most notably that a piece
    of software must be \textbf{repeatedly tested over and over on all available device/ software
    configurations}.

    \medskip
    Additionally, the standard mobile environment does not provide any support for intercepting
    system events (the standard Java edition has a \texttt{java.awt.Robot} class for this).
    
    \medskip
    These issues made us think if it is at all possible to automate capture-replay
    tests in the J2ME environment.
    }
\end{frame}

\begin{frame}
    \frametitle{Related work}

	\begin{itemize}
        \item J2ME Unit
        \item Sony-Ericsson Mobile JUnit
        \item Motorola Gatling
        \item CLDC Unit
        \item IBM Rational Test RT
        \item Research In Motion -- BlackBerry Fledge emulator
	\end{itemize}
    
    \note{
    We did a literature search as well as search for commercial products that would
    offer the functionality we required. Because of limited time I am not able
    to provide you with a full list of features of these products, but it should
    suffice to say that none of them fully implemented what we required. The most
    ``advanced'' solutions -- such as IBM's or BlackBerry's -- are still quite limited.
    For example, they run on software emulators (not on real devices) or are bound 
    to a specific environment (as it is the case with BlackBerry).
    }
\end{frame}

\begin{frame}
    \frametitle{Scope of work}

	\begin{itemize}
        \item Design an environment for implementing acceptance, integration and 
        regression tests for business applications in Java ME.
        \item Possibly automatic capture-replay.
        \item Should work on both actual devices and emulators ({\color{red}\bfseries !}).
        \item Test scripts should be easily interpreted and modified/ 
        maintained by human operators.
	\end{itemize}
    
    \note{
    We set up several goals which we wanted to achieve, outlined on the slide.
    
    Note that while this project is very close to a technical/ engineering one, 
    it is absolutely not trivial to solve. Our major goal was to determine
    if it is at all possible to design and implement an elegant capture-replay solution.
    }
\end{frame}

\section{Solution} % -------------------------------------------------------------------------------

\subsection[Code injection]{Dynamic code injection} % ----------------------------------------------

\begin{frame}
    \frametitle{Dynamic code injection}

	\begin{itemize}
        \item One possible solution for dealing with the mobile environment's constraints.
        \item Intercepting \emph{events} and injecting our custom \emph{proxies}
            at several \emph{injection points}.
	\item User interaction simulator.
	\end{itemize}

    \note{
    The only solution that seemed to solve our goals in practice was to use dynamic code injection.
    
    Code injection/ manipulation is quite common (in OR/M systems, for example). 
    Our tool examines the code of a mobile application and searches for code fragments we would like to 
    intercept. Then, it changes these fragments appropriately by adding (injecting) additional code. 
    The whole process takes place on precompiled Java bytecode, but on the
    following slides I will demonstrate a few examples using the source code for clarity.
    }
\end{frame}

\begin{frame}[containsverbatim,plain]
    \rbcaption{Code injection example -- capturing \texttt{startApp} event.}

\begin{javablock}
public final class MyMidlet extends MIDlet {
    protected void startApp() 
        throws MIDletStateChangeException {
        // original code
    }
...
\end{javablock}

\begin{javablock}
public final class MyMidlet extends MIDlet {
    protected void startApp() 
        throws MIDletStateChangeException {
        // record: before-start-event
        try {
            this.orig$startApp();
            // record: after-start-event
        } catch (Throwable t) {
            // record: start-exception-event
        }
    }

    private void orig$startApp()
        throws MIDletStateChangeException {
        // original code
    }
...
\end{javablock}

    \note{
    The first example demonstrates intercepting the start event of a mobile application. The code snippet on top
    of the screen shows a class that extends the MIDlet class (base class for all mobile applications in Java,
    where the initialization code is typically placed).

    The code snippet on the bottom shows the modified class, where original code from the startApp() method was 
    moved to new method named \texttt{orig.startApp()} and the entire block of previous code is wrapped in 
    a \texttt{try-catch} statement. 
    
    Similar to the above, we are able to intercept all events related to starting, stopping and exceptions handling
    in a mobile application.
    }
\end{frame}

\begin{frame}[containsverbatim,plain]
    \rbcaption{Code injection example -- capturing key events.}

\begin{javablock}
public class Class1 implements CommandListener {
...
	public void commandAction(Command c, Displayable d) {
		// Handling command event
	}
...
}
\end{javablock}

\begin{javablock}
public class Class1 implements CommandListener {
...
	public void commandAction(Command c, Displayable d) {
		RobotMERecorder.getRecorderInstance().commandInvoked(c, d);
		// Handling command event
	}
...
}
\end{javablock}

    \note{
    On this slide we demonstrate how to intercept the events related to key pressing --
    we basically inject custom code at the beginning of each \texttt{commandAction}
    method.

    This way we can intercept and record all user interaction with the mobile application.
    }
\end{frame}

\begin{frame}[containsverbatim,plain]
    \rbcaption{Code injection example -- capturing widgets creation.}

\begin{javablock}
final TextBox textBox = new TextBox("Title", 
  "Hello world", 100, TextField.ANY);
textBox.addCommand(CMD_EXIT);
textBox.addCommand(CMD_OK);

Display.getDisplay(this).setCurrent(textBox);
\end{javablock}

\begin{javablock}
final TextBox textBox = new TextBox("Title", 
  "Hello world", 100, TextField.ANY);

textBox.addCommand(CMD_EXIT);
RobotMERecorder.getRecorderInstance()
  .cmdAddedToDisplayable(CMD_EXIT, textBox);

textBox.addCommand(CMD_OK);
RobotMERecorder.getRecorderInstance()
  .cmdAddedToDisplayable(CMD_OK, textBox);

Display.getDisplay(this).setCurrent(textBox);
RobotMERecorder.getRecorderInstance().setCurrentDisplayable(textBox);
\end{javablock}

    \note{
    The last example shows how to intercept events related to creation and display of various GUI
    elements. These events are essential in order to correctly trace and record the interaction 
    between the user and the application. As the example shows, we again inject some
    additional code after identified parts of the examined application code.
    }

\end{frame}

\subsection{Testing lifecycle} % -------------------------------------------------------------------

\begin{frame}
    \frametitle{Record/replay cycle}

    \begin{center}\large
        RECORD\\
        $\Downarrow$\\
        REPLAY $\Leftrightarrow$ MAINTAIN
    \end{center}
    
    \note{
    On the three following slides I would like to present the phases that we distinguish
    in the record/replay cycle.
    }
\end{frame}

\begin{frame}
    \frametitle{Phase 1: Recording}

	\begin{itemize}
        \item Intercepting user events (key press, pointer events, text edit).
        \item Intercepting environment events (sounds, screen changes, Internet access, SMS).
        \item Transferring to remote console (through GPRS, bluetooth, IRDA or serial port).
	\end{itemize}
	
    \note{
    The first phase -- the recording phase -- is the process of capturing every event between the software
    and the user, not necessarily events directly initiated by the user, but also sounds, screen changes etc.
    
    The recorded events are transferred through an effective binary protocol to the remote console. The remote console is a 
    server application that is designed to receive data transferred by the recording mobile application
    and store it in a file.
    }
\end{frame}

\begin{frame}
    \frametitle{Phase 2: Replaying}

	\begin{itemize}
        \item Simulating user events (key press, pointer events, text edit).
        \item Assertions, injection points: intercepting environment events (sounds, screen changes, internet access, SMS).
        \item Transferring to remote console (through GPRS, bluetooth, IRDA or serial port).
	\end{itemize}

    \note{
    The second phase -- replay -- is a process of repeating previously recorded user interaction on the actual device (or simulator)
    and verifying if the application responds similarly to when the recording was performed. This verification process is also called 
    'assertion checking'. Similarly to the recording phase, also this time all the intercepted events are 
    transferred back to the server for further investigation in case something went went wrong during replay.
    }
\end{frame}

\begin{frame}[containsverbatim]
    \frametitle{Phase 3: Test maintenance}

    Maintenance through \emph{human-comprehensible test scripts}.

    \bigskip
\begin{xmlblock}
<scenario>
  <event timestamp="1000">
      <displayable-changed title="Hello screen" type="TEXTBOX" />
  </event>
    
  <event timestamp="2000">
      <command cmdLabel="Start app" displayableTitle="Hello screen" />
  </event>
    
  <event timestamp="3000">        
      <textbox-modification assertion="true" strongAssertion="true" 
                            string="I like testing" />
  </event>
</scenario>
\end{xmlblock}

    \note{
    Test maintenance is a phase parallel to recording and replay of test scripts. 

    For performance reasons, the recorded scripts are stored in a compact binary format. 
    To enable its maintanance, we created a converter that transforms the binary format
    to and from a human-comprehensible XML script.

    This way we can edit the test scripts to, for example, skip some unnecessary assertions, or the opposite --
    add additional assortions that the recording tool was not able to capture (like some additional time constraints).
    }
\end{frame}

\begin{frame}
    \frametitle{Example assertions}

	\begin{itemize}
        \item Text on the screen changed
	\item Screen changed
	\item Sound emission
	\item Access to remote server
	\item Access to bluetooth receiver
	\item SMS send
	\item Data stored or read from persistent storage
	\end{itemize}

    \note{
    We have distinguished several types of assertions, as shown on the slide. 
    We can determine and track elements of the graphical
    user interface present on the screen. We can also check the status of network communication (open connections)
    and local transfers to and from a persistent memory (record store).
    
    These assertions allow to define the expected behavior
    of the application under test pretty closely.
    }
\end{frame}


\subsection[Prototype]{Proof-of-concept prototype} % -----------------------------------------------

\begin{frame}[plain]
	\frametitle{Screenshots from test recording session.}

    \lbcaption{Server console.}
    \rbcaption{Emulator window.}

    \fxygraphics{12.4}{0.1}{right}{bottom}{height=8.5cm}{figures/1b}
    \fxygraphics{0.2}{8.4}{left}{top}{width=12.4cm}{figures/2}
    
    \note{
    At the end as kind of proof that our idea is working in reality
    I would like to present you simple live demonstration of the
    created tool. Demonstrating application is a questionnaire with a few questions, some
    with textual answers, with date, level and checkbox answers.
    
    I have prepared recorded script, so I will demonstrate you only replaying phase
    of our tool. First lets run a remote console application. And then lets run
    a simulator. As you can see...
    }
\end{frame}b


\section*{Summary} % -------------------------------------------------------------------------------

\begin{frame}[plain]
    \frametitle{Summary}
    
    \begin{itemize}
        \item Useful testing framework for J2ME environment \emph{does not exist}.
        \item Solution: \emph{dynamic code injection}.
        \item For \emph{developers} and \emph{clients}.
    \end{itemize}
    
    \note{
    Summarizing: I presented you testing framework for J2ME platform that in our opinion
    could be very useful for mobile business applications vendors, and the framework
    that has no real competitors on the market till now. Which is also very important it
    could be used by both technical users (developers) and business users (clients).
    }
\end{frame}

\begin{frame}[plain]
	\begin{center}
	Thank you for your attention.
	\end{center}
\end{frame}

\end{document}

